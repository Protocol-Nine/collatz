\documentclass[letterpaper]{article}

% Package Imports and Configurations
\usepackage{graphicx} %LaTeX package to import graphics
\graphicspath{{images/}} %configuring the graphicx package
\usepackage{amsmath}
\usepackage{array}
\usepackage{geometry}
\geometry{letterpaper, portrait, margin=0.75in}
\usepackage{multicol}
\usepackage{wrapfig}

%Paper Information
\title{Entropy Makes the Collatz Sequence Go Down}
\author{Bradley Berg\\Edited by Erik Slader}

\begin{document}

\maketitle

\begin{multicols}{2}

    \begin{abstract}
        We look at the Collatz Sequence from an information theory perspective to lay out its underlying computational mechanics. The mechanisms are similar to those used in pseudo random number generators and one way hashes.
        An individual run is divided into three phases.  In the first phase the influence of the seed runs its course after information contained in the initial seed is lost.  Values are randomized in the second phase.  They follow the statistical model where the average gain is just over 0.866; causing them to decline.  The third phase begins once a value goes below the seed; providing that the series is not circular.  At this point we know the run will terminate at one.
        An equivalent restatement of the Collatz sequence steps through alternating chains of even and odd values.  This variation constitutes a pseudo random number generator.  The operations used to scramble values are unbiased which results in an even distibution of ones and zeros.  The entropy of this mechanism is high so that in the second phase values are fairly randomized.
    \end{abstract}

    \section{Introduction}


    The computational mechanics of the Collatz sequence are analyzed to determine the odds of taking an even or odd step. With the following definition of the sequence we show the the average odds of taking either step are even. In this case statistically the sequence will on average decline and eventually terminate. This variation is often referred to as the Syracuse sequence.

    \begin{align*}
        \text{N is Even:} & \quad N' = \frac{N}{2}      \\
        \text{N is Odd:}  & \quad N' = \frac{3N + 1}{2}
    \end{align*}

    The gain of a each transition is its Output divided by the Input (N' / N). As N gets larger in the odd transition the "+ 1" term quickly becomes insignificant. To compute the gain for the odd transition in the limit we can safely drop the "+1"term.

    \begin{center}
        \begin{tabular}{l c c}
                       & Output / Input      & Gain \\
            N is Even: & $\frac{N}{2N}$      & 0.5  \\
                       &                     &      \\
            N is Odd:  & $\frac{3N + 1}{2N}$ & 1.5
        \end{tabular}
    \end{center}

    The total gain of a series of transitions is the product of the gains for each transition. Based on this the average gain of a sequence depends on the probability of taking either an odd or even path.

    \begin{align*}
        G_s = 1.5^{p_{\text{odd}}} \cdot 0.5^{p_{\text{even}}}
    \end{align*}

    The choice of which path is taken is determined by the low order bit of the input value. If the sequence should produces uniformly randomized values then the chances of taking either transitions is 50:50. This implies the low bit would need to be uniformally random over the sequence. The average gain of a uniformly randomized sequence is then:

    \begin{align*}
        \text{Average transaction gain}    & = \langle \text{transaction gain} \rangle^{p(\langle \text{transaction} \rangle)} \\
        \text{Average odd gain}            & = 1.5^{0.5} = 1.22474                                                             \\
        \text{Average even gain}           & = 0.5^{0.5} = 0.70711                                                             \\
        \text{Average sequence gain: } G_a & = 1.22474 \cdot 0.70711 = 0.86603+
    \end{align*}

    The statistical average gain in each step is less than one so on average the sequence declines. However, the gain in a single run for a given Seed can vary significantly. There could potentially still be a series where the the total gain indefinitely exceeds one and never terminates.

    \begin{align*}
        \text{Breakeven Gain}                   & = 1.5^p \cdot 0.5^{1-p} = 1                   \\
        \ln\left( 1.5^p \cdot 0.5^{1-p} \right) & = \ln(1) = 0                                  \\
                                                & = \ln(1.5^p) + \ln(0.5^{1-p}) = 0             \\
                                                & = p \cdot \ln(1.5) + (1-p) \cdot \ln(0.5) = 0 \\
        p \cdot 0.40546                         & = -(1-p) \cdot (-0.69315)                     \\
        \frac{0.40546}{-0.69315}                & = \frac{-(1-p)}{p}                            \\
        -0.58497                                & = -\frac{1-p}{p}                              \\
        -0.58497 - 1                            & = -\frac{1}{p}                                \\
        p                                       & = \frac{1}{1.58497} \approx 0.63093
    \end{align*}
    \begin{align*}
        \text{Gain} & = 1.5^{0.63} \cdot 0.5^{0.37} \approx 0.99898 \quad \text{Under breaking even} \\
        \\
        \text{Gain} & = 1.5^{0.64} \cdot 0.5^{0.36} \approx 1.01001 \quad \text{Over breaking even}
    \end{align*}

    For any series of values to continually increase and never terminate it would have to sustain an average gain over one. To break even odd transitions would need to occur about 64\% of the time. They would need to be applied over 1.7 times more than evens; which is substantially skewed. It remains to be shown that the sequence does not intrinsically favor odd transitions.

    \subsection{Even and Odd Chains}

    Consecutive iterations of the same kind of transition in a run form a chain. Even chains start with an even seed value that in binary have one or more trailing zeroes. After applying the transitions in an even chain the result simply has the low order zeros removed.

    Odd chains consume an odd input and have multiple odd intermediate values. Eventually an Odd chain transitions to an even number. The number of consecutive low order one bits determines the chain length. For example, an input of 19 is a binary 10011 so the subsequent chain has two Odd transitions: 19 $\to$ 29 $\to$ 44

    Let \( k \) be the number of low order one bits and \( j \) is the input value with the low one bits removed plus 1.

    The input to an odd chain has the form: \( j \cdot 2^k - 1 \)

    The output of the chain simplifies to: \( j \cdot 3^k - 1 \)

    \begin{align*}
        N_1 & = \frac{3N + 1}{2}                 &  & \text{First transition}                \\
            & = \frac{3(j \cdot 2^k - 1) + 1}{2} &  & \text{Substitute } N = j \cdot 2^k - 1 \\
            & = \frac{3j \cdot 2^k - 3 + 1}{2}                                               \\
            & = \frac{3j \cdot 2^k - 2}{2}                                                   \\
            & = 3^1 \cdot j \cdot 2^{k-1} - 1
    \end{align*}

    \begin{align*}
        N_{i+1} & = \frac{3 \left(3^i \cdot j \cdot 2^{k-i} - 1\right) + 1}{2} &  & \text{Subsequent transitions} \\
                & = \frac{9 \cdot j \cdot 2^{k-i} - 3 + 1}{2}                                                     \\
                & = \frac{9 \cdot j \cdot 2^{k-i} - 2}{2}                                                         \\
                & = 3^{i+1} \cdot j \cdot 2^{k-i-1} - 1                                                           \\
        N_k     & = 3^k \cdot j - 1                                            &  & \text{Odd chain output}
    \end{align*}

    Each run of the Collatz sequence will have segments with alternating even and odd chains. For reference, here are the first few chains for the series beginning with a seed of 27.

    \noindent
    \begin{minipage}{\columnwidth}
        \setlength{\tabcolsep}{.35\tabcolsep}
        \centering
        \begin{tabular}{|l|c|c|c|c|r|}
            \hline
            \textbf{Syracuse Sequence}                      & \textbf{Even} & \textbf{Odd} & \textbf{j} & \textbf{k} & \textbf{Base 2} \\
            \hline
            27 $\to$ 41 $\to$ 124 $\to$ 62                  &               & 27 $\to$ 62  & 7          & 2          & 110\_11         \\
            \hline
            $\to$ 31                                        & $\to$ 31      &              & 31         & 1          & 11111\_0        \\
            \hline
            $\to$ 47 $\to$ 71 $\to$ 107 $\to$ 161 $\to$ 242 &               & $\to$ 242    & 1          & 5          & 0\_11111        \\
            \hline
            $\to$ 121                                       & $\to$ 121     &              & 121        & 1          & 111100\_1       \\
            \hline
            $\to$ 364 $\to$ 182                             &               & $\to$ 182    & 61         & 1          & 1011011\_0      \\
            \hline
            $\to$ 91                                        & $\to$ 91      &              & 91         & 1          & 10110\_11       \\
            \hline
            $\to$ 137 $\to$ 206                             &               & $\to$ 206    & 23         & 2          & 1100111\_0      \\
            \hline
            $\to$ 103                                       & $\to$ 103     &              & 103        & 1          & 1100\_111       \\
            \hline
            $\to$ 155 $\to$ 233 $\to$ 350                   &               & $\to$ 350    & 13         & 3          & 10101111\_0     \\
            \hline
            $\to$ 175                                       & $\to$ 175     &              & 175        & 1          & 1010\_1111      \\
            \hline
            $\to$ 263 $\to$ 395 $\to$ 593 $\to$ 890         &               & $\to$ 890    & 11         & 4          & 110111101\_0    \\
            \hline
        \end{tabular}
    \end{minipage}

    \subsection{Combining Even And Odd Chains}

    Each run alternates between even and odd chains. We represent this algebraically by merging both into a single step. This gives us a series defined as a single transition. Since all intermediate values using this combined definition are even, an initial odd value needs to first transition one step using the "3n + 1" rule to reach the first even number.

    Every input has the binary form: [$j$] [ones] [zeros]
    \[
        N = ([j + 1] \cdot 2^{k_o} - 1) \cdot 2^{k_z} \rightarrow [j + 1] \cdot 3^{k_o} - 1
    \]
    where:
    \begin{itemize}
        \item $k_z$ - number of trailing zeros in $N$
        \item $k_o$ - number of the next higher set of ones in $N$
        \item $j$ - $N$ shifted right by $k_z + k_o$ bits: $\frac{N}{2^{k_z + k_o}}$
    \end{itemize}

    Using the previous example, initially we transition 27 to 82. From there the next few steps are:

    \noindent
    \begin{minipage}{\columnwidth}
        \setlength{\tabcolsep}{.35\tabcolsep}
        \centering
        \begin{tabular}{|l|c|c|c|r|}
            \hline
                                             & \textbf{j} & \textbf{ko} & \textbf{kz} & \textbf{binary} \\
            \hline
            82 $\to$ 41  $\to$  124 $\to$ 62 & 20         & 1           & 2           & 10100\_10       \\
            \hline
            $\to$ 31  $\to$  484 $\to$ 242   & 0          & 5           & 1           & 0\_111110       \\
            \hline
            $\to$ 121 $\to$  364 $\to$ 182   & 60         & 1           & 1           & 111100\_10      \\
            \hline
            $\to$  91 $\to$  274 $\to$ 206   & 22         & 2           & 1           & 10110\_110      \\
            \hline
            $\to$ 103 $\to$  310 $\to$ 466   & 12         & 3           & 1           & 1100\_1110      \\
            \hline
            $\to$ 233 $\to$  700 $\to$ 350   & 116        & 1           & 1           & 1110100\_10     \\
            \hline
            $\to$ 593 $\to$ 1780 $\to$ 890   & 10         & 4           & 1           & 1010\_11110     \\
            \hline
        \end{tabular}
    \end{minipage}

    \section{Sequence Entropy}

    Statistical averages only hold when the odds are fair. In this section, we show why the dice are not loaded. Shannon entropy is a measure of information denoting the level of uncertainty about the possible outcomes of a random variable \cite{1}.

    \[
        H = -p0 \cdot \log_2(p0) - p1 \cdot \log_2(p1)
    \]

    where:
    \begin{itemize}
        \item $p0$ is the probability a bit is zero.
        \item $p1$ is the probability a bit is one ($1 - p0$).
    \end{itemize}

    A set of coin tosses has $p0 = p1 = 0.5$ so its entropy is 1; totally random. When looking at the entropy of bits in a number then $p0$ is the percentage of zero bits. For the binary number, $1010\_1111$, $p0$ is $0.25$ ($H = 0.811$). Strings of all ones or zeros have no entropy ($H = 0$). For a binary number, we are measuring the bits in a number horizontally.

    Bits in a series of numbers have two dimensions - horizontal bits in each individual value and vertical bits over the duration of the series. We can also measure entropy vertically over a number series. That means we can observe a select bit position in each value as the series progresses.

    For Collatz, the low-order bit is of interest because it determines if a number is odd or even. In turn, that determines which transition to take. When the entropy of the low-order bit is high, then on average there are nearly as many even transitions as odds.

    Each kind of chain takes a value where the low-order bits are a string of zeros or ones and either removes or replaces them. Even inputs remove low-order zeros. The expression for odd chain inputs has a $2^k$ term that transitions to a $3^k$ term.

    Since strings of zeros have no entropy and the $j$ term has positive entropy, entropy increases each time an even transition is applied. In odd transitions, entropy is also increased by removing the repeated ones and again by scrambling the remaining bits. The upper bits, $j$, are scrambled by multiplying $j$ by a power of 3. As a run progresses, this increase in entropy randomizes the values. The number of odd and even transitions balance out driving the sequence downward and eventually forcing it to terminate.

    \subsection{Losing Information}
    A Seed can be contrived to produce a run of any desired length. The longer the run, the larger the seed has to be in order to contain enough information to influence the desired outcome. Initially, as a run progresses, the information contained in the Seed is lost. When there are two possible ways to reach a value in a run we lose the information about which path was taken to reach it \cite{2}.

    Odd numbers always transition 3n + 1 to even numbers, so an odd value can only be reached from one even value. However, certain even numbers can be reached from either an odd or even transition. For example, an output of 16 can be reached from either 32 or 5:

    \[
        \begin{aligned}
             & 32 \rightarrow 32 / 2 = 16     \\
             & 5 \rightarrow 3\cdot5 + 1 = 16
        \end{aligned}
    \]

    Whenever transitioning to an even value such that $(\text{even} \% 3 = 1)$, then the previous value could have been either:

    \[
        2 \cdot \text{even or } (\text{even} - 1) / 3
    \]

    For Collatz, a bit of information contained in the seed is lost each time one of these select even numbers is reached. After all bits in the seed are scrubbed, this initial phase is complete. Any attempt to contrive a Seed to skew results can only directly affect values during this phase.

    One-way hash functions rely on this concept of lost information \cite{3}. Secure hashes have many ways to reach each hashed value. This is how passwords are encoded and used for authentication. Using this metaphor, you can think of the Seed as a password and the Collatz sequence as a trivial one-way hash schema used to mask it.

    \subsection{RandomizationPhase}

    This next phase is key as this is where the sequence runs below the Seed. The sequence is rewritten as a pseudo random number (PRNG) generator. Hastad et al. (1999) \cite{4} show that any one-way hash can be used to create a PRNG. Uniformly randomized values eventually trend towards their average. In turn, this drives transitions towards their average gain. In the introduction, we showed that Collatz has an average gain of $0.86603$, eventually driving the series below the Seed value.

    We'll be using the combined series from section 1.2 for each randomization step. The value, $j$, is always even so the $[j + 1]$ term will simply set the low order bit to one as there is no carry. Also, the product will have an odd result so that decrementing by $1$ will likewise just clear the low order bit.

    \[
        \text{Input: } ([j + 1] \cdot 2^{k_o} - 1) \cdot 2^{k_z}
    \]

    \[
        \text{Result: } [j + 1] \cdot 3^{k_o} - 1 = [j \oplus 1] \cdot 3^{k_o} \oplus 1
    \]



    \begin{figure*}
        \includegraphics[width=\textwidth]{collatz_even}
    \end{figure*}

    \newpage

    In this next example, the top line has steps for a randomization phase that begins with $647$. Calculations for each combined transition ($1942, 2186, 1640, 308, 116$) are shown in binary:

    \noindent
    \begin{minipage}{\columnwidth}
        \setlength{\tabcolsep}{.35\tabcolsep}
        \centering
        \begin{tabular}{|l|r|r|r|r|}
            \hline
                           & \textbf{$1942$} & \textbf{$2186$} & \textbf{$1640$} & \textbf{$116$} \\
            \hline
            Input          & 11110010\_110   & 1000100010\_10  & 1100110\_1000   & 100110\_100    \\
            \hline
            Shift Right    & 11110010        & 1000100010      & 1100110         & 100110         \\
            \hline
            Xor 1          & 11110011        & 1000100011      & 1100111         & 100111         \\
            \hline
            Times $3^{ko}$ & 100010001011    & 11001101001     & 100110101       & 1110101        \\
            \hline
            Xor 1          & 1000100010\_10  & 1100110\_1000   & 100110\_100     & 1110100        \\
            \hline
        \end{tabular}
    \end{minipage}

    A pseudo-random number generator repeatedly applies a function to produce a series of values. In order to produce uniformly random numbers, operators cannot be biased towards producing either more ones or zeros. In a uniform sequence, the entropy will be one. If it is not uniform, the bias will show up in the operators.

    \textbf{Select}

    If you remove some low-order bits of a random number, the remaining part will still be random. Using the upper bits from the input still gives random values. However, the way the value is split, the selected upper value will be even. The low-order bit is zero, and only the other bits are a randomized portion.

    \textbf{Logical Exclusive Or 1}

    The first Exclusive Or sets the low bit of the selected region. This is balanced out by clearing it in the final step with another Exclusive.

    \textbf{Product}

    The product of a random variable by a constant is also random, but with a larger gap between them. Multiplying random numbers from 1 to 10 by 3 yields random numbers from 3 to 30. They simply have a gap of 3 between them instead of 1.

    The product used to scramble values is equivalent to repeated sums of the input. The following table shows all combinations for the three inputs (A, B, Carry In) and the two outputs (Sum, Carry Out). It also shows changes (Exclusive Or) between the sum and inputs A and B:

    \noindent
    \begin{minipage}{\columnwidth}
        \setlength{\tabcolsep}{.35\tabcolsep}
        \centering
        \begin{tabular}{|c c c|c c|c c|}
            \hline
            \textbf{A} & \textbf{B} & \textbf{Carry} & \textbf{Sum} & \textbf{Carry} & \textbf{A $\oplus$} & \textbf{B $\oplus$} \\
                       &            & \textbf{In}    &              & \textbf{Out}   & \textbf{Sum}        & \textbf{Sum}        \\
            \hline
            0          & 0          & 0              & 0            & 0              & 0                   & 0                   \\
            0          & 0          & 1              & 1            & 0              & 1                   & 1                   \\
            0          & 1          & 0              & 1            & 0              & 1                   & 0                   \\
            0          & 1          & 1              & 0            & 1              & 0                   & 1                   \\
            1          & 0          & 0              & 1            & 0              & 0                   & 1                   \\
            1          & 0          & 1              & 0            & 1              & 1                   & 0                   \\
            1          & 1          & 0              & 0            & 1              & 1                   & 1                   \\
            1          & 1          & 1              & 1            & 1              & 0                   & 0                   \\
            \hline
        \end{tabular}
    \end{minipage}

    Input bits A and B are vertically aligned and are altered by addition. Carries are applied horizontally and propagate to higher order bits. This way, bits in both directions become scrambled.

    Note that all the columns in the table are different. This shows how bits are scrambled to produce randomized results. Also note that all columns contain 4 zeros and 4 ones. This balance produces results that are unbiased towards either zero or one bits. The end result is a series of uniformly distributed pseudo-random numbers.

    Outputs in any individual series depend on the values $k_z$ and $k_o$. The more random they are, the more random the series. The $k$ values measure the width of a horizontal subset of bits in each value. Pseudo-random number generators that conflate operations on horizontal and vertical sets of bits rely on the independence of these orthogonal values.

    The repeated zeros and upwardly ones in the lowest bits that might have low entropy are continuously removed and replaced with scrambled bits. This creates a self-regulating system that continuously randomizes the lower bits. Since those bits control the selection operation in the next round.

    When runs have uniformly random values, then revisiting the Syracuse sequence, the average number of even and odd transitions will balance out. In turn, this causes the run to decrease since the average gain is less than one. If the sequence was not uniformly random, then we would see bias amongst the arithmetic operations used in each round.

    Examples where seeds produce long runs will have highly skewed values in the first phase, but that cannot be sustained. As values become more randomized and the series progresses, they will trend towards average results. With coin tossing, even if you get lucky and call the results of several coin tosses, your luck will run out in the long run.

    \subsection{Reduction To One}

    The previous randomization phase leaves us with a value of \( N \) that is below the seed. It is well understood that once this happens, we know the series will eventually terminate at one. Firstly, we know all values below some arbitrary small number \( M \) (say 10) transition to one.

    Next, starting with the next higher Seed, \( M + 1 \), we transition until it reaches \( M \) or less. Since we already know Seeds of \( M \) or less will reach one, by induction, once a series goes below its Seed we know it will reach one. This is why even Seeds are uninteresting as they immediately decline.

    When measuring the length of a run, including this phase is not useful and can distort any result. When winding down as numbers get smaller, they can become more irregular. Instead of defining the run length as the number of steps to one, use only the number of odd steps until the series goes below the Seed. It is usually more practical to count only odd steps because it corresponds to the number of terms in the algebraic expansion of a run.

    \subsection{Observed Entropy}

    The first few values will have lower entropy until enough bits are included to average out. In the Introduction, we've shown that to sustain an infinite run there needs to be 64\% or more ones. This gives an entropy of:

    \[ H = -0.36 \cdot \log_2(0.36) - 0.64 \cdot \log_2(0.64) = 0.94268 \]

    Individual runs will typically have some jitter since we are performing discrete computations. There will be higher entropy at the end of very long runs, which are rare. In the first phase, long runs will be skewed towards more odd steps to make the values grow larger up front. Short runs where evens dominate won't even make it to the randomization phase.

    To compute the entropy, the low-order zero bit is discarded as it is fixed. Also, since the values have a variable width, the uppermost bit where one is also discarded. This differs from practical PRNGs where the values have a fixed width.

    To see the randomization in action, this trace lists entropy in the first two phases. Entropy is computed using the accumulated number of ones and zeros in the run. The counts of ones and zeros are reset at the start of the Randomization phase so that those computations are completely separate. Even in the Information Loss phase, entropy is well above the 0.94268 bound right out of the gate. The computed length of the Information Loss phase is quite conservative.

    \noindent
    \begin{minipage}{\columnwidth}
        \setlength{\tabcolsep}{.35\tabcolsep}
        \centering
        $$\textbf{Seed} = 4\_50449\_75045\_09599 = 10\_00\text{d}1\_0\text{da}5\_\text{de}9\text{f}_{base 16}$$
        \begin{tabular}{|r|r|r|r|c|}
            \hline
            \textbf{Step} & \textbf{Entropy} & \textbf{Ones} & \textbf{Zeros} & \textbf{Notes} \\
            \hline
            1                  & 0.99750          & 24            & 27             & Information    \\
            2                  & 0.99993          & 52            & 51             & Loss           \\
            3                  & 0.99988          & 77            & 79             & phase 1        \\
            4                  & 0.99998          & 105           & 104            &                \\
            5                  & 0.99934          & 136           & 128            &                \\
            6                  & 0.99965          & 163           & 156            &                \\
            7                  & 0.99950          & 194           & 184            &                \\
            8                  & 0.99986          & 222           & 216            &                \\
            9                  & 0.99999          & 250           & 248            &                \\
            10                 & 0.99994          & 281           & 276            &                \\
            11                 & 0.99973          & 314           & 302            &                \\
            12                 & 0.99984          & 342           & 332            &                \\
            13                 & 0.99993          & 369           & 362            &                \\
            14                 & 0.99966          & 402           & 385            &                \\
            15                 & 0.99983          & 428           & 415            &                \\
            16                 & 0.99989          & 456           & 445            &                \\
            17                 & 0.99982          & 487           & 472            &                \\
            18                 & 0.99975          & 518           & 499            &                \\
            19                 & 0.99988          & 545           & 531            &                \\
            \hline
            20                 & 0.99920          & 31            & 29             & Randomization       \\
            30                 & 1.00000          & 321           & 321            & phase 2      \\
            40                 & 0.99978          & 606           & 585            &        \\
            50                 & 0.99966          & 899           & 861            &                \\
            60                 & 0.99983          & 1192          & 1156           &                \\
            70                 & 0.99969          & 1489          & 1429           &                \\
            80                 & 0.99996          & 1770          & 1744           &                \\
            90                 & 0.99963          & 2105          & 2012           &                \\
            100                & 0.99967          & 2410          & 2309           &                \\
            110                & 0.99876          & 2772          & 2551           &                \\
            120                & 0.99828          & 3105          & 2816           &                \\
            130                & 0.99890          & 3387          & 3133           &                \\
            \hline
        \end{tabular}
    \end{minipage}

    If the hash function had a bias, it would show up by running it over many consecutive numbers. Here the hash was run over a million consecutive even numbers. This next chart shows the cumulative entropy of the resulting values, which is very near one as expected.

    \noindent
    \begin{minipage}{\columnwidth}
        \setlength{\tabcolsep}{.35\tabcolsep}
        \centering
        \begin{tabular}{|c|c|c|c|}
            \hline
            \textbf{Iteration} & \textbf{Entropy} & \textbf{Ones} & \textbf{Zeros} \\
            \hline
            50,000             & 0.99350          & 475,206       & 574,794        \\
            100,000            & 0.99721          & 984,678       & 1,115,322      \\
            150,000            & 0.99886          & 1,512,405     & 1,637,595      \\
            200,000            & 0.99966          & 2,054,339     & 2,145,661      \\
            250,000            & 0.99984          & 2,585,879     & 2,664,121      \\
            300,000            & 0.99970          & 3,086,254     & 3,213,746      \\
            350,000            & 0.99987          & 3,625,941     & 3,724,059      \\
            400,000            & 0.99970          & 4,113,665     & 4,286,335      \\
            450,000            & 0.99976          & 4,638,681     & 4,811,319      \\
            500,000            & 0.99983          & 5,168,585     & 5,331,415      \\
            550,000            & 0.99991          & 5,711,156     & 5,838,844      \\
            600,000            & 0.99996          & 6,255,089     & 6,344,911      \\
            650,000            & 1.00000          & 6,816,718     & 6,833,282      \\
            700,000            & 0.99994          & 7,415,041     & 7,284,959      \\
            750,000            & 1.00000          & 7,882,886     & 7,867,114      \\
            800,000            & 1.00000          & 8,392,173     & 8,407,827      \\
            850,000            & 1.00000          & 8,925,576     & 8,924,424      \\
            900,000            & 1.00000          & 9,428,185     & 9,471,815      \\
            950,000            & 0.99999          & 9,944,542     & 10,005,458     \\
            1,000,000          & 0.99999          & 10,466,775    & 10,533,225     \\
            1,048,576          & 1.00000          & 11,012,891    & 11,007,205     \\
            \hline
        \end{tabular}
    \end{minipage}

    This next chart shows a contorted sequence where some bits are artificially forced to one. This shows how skewed the operators have to get in order for entropy to drop below 0.94268, where it would increase indefinitely. 11 out of 20 bits are needed to be forced to one to sufficiently skew the result.

    \noindent
    \begin{minipage}{\columnwidth}
        \setlength{\tabcolsep}{.35\tabcolsep}
        \centering
        \begin{tabular}{|c|c|c|c|}
            \hline
            \textbf{Value Of}    & \textbf{Final}   &               &                \\
            \textbf{Forced Bits} & \textbf{Entropy} & \textbf{Ones} & \textbf{Zeros} \\
            \hline
            0                    & 1.00000          & 11,012,891    & 11,007,205     \\
            10,000               & 0.99995          & 11,102,306    & 10,917,790     \\
            11,000               & 0.99971          & 11,229,280    & 10,790,816     \\
            11,100               & 0.99931          & 11,350,975    & 10,669,121     \\
            11,500               & 0.99916          & 11,385,687    & 10,634,409     \\
            15,500               & 0.99892          & 11,435,329    & 10,584,767     \\
            55,500               & 0.99703          & 11,716,482    & 10,303,614     \\
            155,500              & 0.99727          & 11,687,638    & 10,332,458     \\
            175,500              & 0.98541          & 12,573,465    & 9,446,631      \\
            177,500              & 0.96792          & 13,323,124    & 8,696,972      \\
            177,700              & 0.95400          & 13,775,529    & 8,244,567      \\
            177,740              & 0.94266          & 14,093,550    & 7,926,546      \\
            \hline
        \end{tabular}
    \end{minipage}

    \section{Conclusion}

    The Collatz sequence incorporates principle mechanisms commonly used to create pseudo random number generators.

    \begin{itemize}
        \item To overcome a contrived Seed, one way hashing smothes out any regularities.
        \item Repeated low order one and zero bits are erased at each step.
        \item A product using independent values randomizes values.
    \end{itemize}
    Any individual run is partitioned into three phases. In the initial phase the Seed value can influence the outcome to produce arbitrarily long runs. After that the series generates randomized values until it goes below the Seed value. From there it is guaranteed to reduce to one unless the series is circular.
    A uniformly randomized series eventually moves towards a statistically average gain. For a Collatz series to sustain an average gain above one would require over 1.7 times more odd transitions than even. This is well above parity. Instead, randomization forces the series to average out and decrease until it inevitably goes below the seed. Once it does that we know it will terminate.
    The random behavior of the Collatz sequence makes it impossible to prove algebraically. Conway[5] showed that a generalization of the 3N + 1 problem is undecidable. Trying to make sense of the values in the series is akin to analysing values produced by a random number generator. The irony is that this randomness is the force that leads to convergence.

    \begin{thebibliography}{9}
        \bibitem{1}
        Behrouz Zolfaghari, Khodakhast Bibak, and Takeshi Koshiba \emph{"The Odyssey of Entropy: Cryptography"}, Entropy 2022, 24(2), 266.%\\https://www.mdpi.com/1099-4300/24/2/266/pdf
        \bibitem{2}
        John C. Baez, Tobias Fritz, Tom Leinster, \emph{"A Characterization of Entropy in Terms of Information Loss"}, Entropy 2011, 13(11), 1945-1957%\\https://mdpi-res.com/d_attachment/entropy/entropy-13-01945/article_deploy/entropy-13-01945-v2.pdf
        \bibitem{3}
        Russell Impagliazzo, Leonid A. Levin, Michael Luby;\emph{"Pseudo-random Generation from one-way functions"}, in David S. Johnson (ed.),Proceedings of the 21st Annual ACM Symposium on Theory of Computing,May 14-17, 1989, Seattle, Washington, USA, {ACM}, pp. 12-24,doi:10.1145/73007.73009, S2CID 18587852%\\https://dl.acm.org/doi/pdf/10.1145/73007.73009
        \bibitem{4}
        Johan Hastad, Russell Impagliazzo, Leonid A. Levin, Michael Luby;\emph{A pseudorandom generator from any one-way function.} Siam Journal on Computation 28(4):1364-1396, 1999.%\\https://epubs.siam.org/doi/10.1137/S0097539793244708
        \bibitem{5}
        John H. Conway, \emph{"Unpredictable iterations"}. Proc. 1972 Number Theory Conf., Univ. Colorado, Boulder. pp. 49-52.
    \end{thebibliography}

\end{multicols}

\appendix

\section*{Appendix}

Here are the entropy values for the randomization phase of some long runs. The Sample Size is the number of values produced by the algorithm for randomization. All of the entropy values are well above 0.94268; the entropy required to produce an infinitely long run.

Entropy is computed from values in each run. Since at the end of each step the values are all even, the low order bit is discarded. Values also have variable widths; unlike values in a practical PRNG. To account for this the upper one bit is also discarded. The entropy is then computed using the total number of ones and zeros in each run. To illustrate this the chart shows the entropy after five steps into the randomization phase.

\noindent\makebox[\textwidth]{
    \begin{tabular}{|c c c c| c c c c|}
        \hline
        \textbf{Run}    &                   &                  & \textbf{Sample} & \textbf{Run} &                       &                  & \textbf{Sample} \\
        \textbf{Length} & \textbf{Seed}     & \textbf{Entropy} & \textbf{Size}   & \textbf{Run} & \textbf{Seed}         & \textbf{Entropy} & \textbf{Size}   \\
        \hline
        200             & 371\_871\_359     & 0.99979          & 68              & 292          & 331\_224\_689\_767    & 0.99971          & 117             \\
        \hline
        201             & 247\_914\_239     & 0.99979          & 69              & 293          & 188\_890\_883\_743    & 0.99946          & 92              \\
        \hline
        202             & 165\_276\_159     & 0.99980          & 69              & 294          & 215\_384\_833\_215    & 0.99991          & 104             \\
        \hline
        203             & 2\_173\_615\_775  & 0.99998          & 61              & 295          & 167\_903\_007\_771    & 0.99952          & 92              \\
        \hline
        204             & 293\_824\_283     & 0.99974          & 69              & 296          & 144\_460\_775\_535    & 0.99995          & 104             \\
        \hline
        205             & 195\_882\_855     & 0.99980          & 70              & 297          & 125\_291\_645\_607    & 0.99938          & 101             \\
        \hline
        206             & 620\_752\_511     & 0.99903          & 62              & 298          & 196\_281\_297\_639    & 0.99975          & 118             \\
        \hline
        207             & 348\_236\_187     & 0.99969          & 70              & 299          & 1\_063\_641\_582\_407 & 0.99992          & 102             \\
        \hline
        208             & 413\_835\_007     & 0.99871          & 62              & 300          & 709\_094\_388\_271    & 0.99993          & 102             \\
        \hline
        209             & 1\_651\_171\_495  & 0.99810          & 68              & 301          & 473\_644\_547\_375    & 0.99953          & 103             \\
        \hline
        210             & 127\_456\_255     & 0.99983          & 95              & 302          & 380\_103\_773\_863    & 0.99979          & 103             \\
        \hline
        211             & 245\_235\_559     & 0.99882          & 65              & 303          & 315\_763\_031\_583    & 0.99956          & 104             \\
        \hline
        212             & 5\_425\_672\_039  & 1.00000          & 71              & 304          & 284\_396\_952\_295    & 0.99929          & 105             \\
        \hline
        213             & 217\_987\_163     & 0.99892          & 66              & 305          & 33\_980\_539\_439     & 0.99951          & 103             \\
        \hline
        214             & 290\_649\_551     & 0.99894          & 66              & 306          & 150\_164\_453\_871    & 0.99984          & 105             \\
        \hline
        215             & 193\_766\_367     & 0.99899          & 67              & 307          & 22\_653\_692\_959     & 0.99946          & 105             \\
        \hline
        216             & 145\_324\_775     & 0.99894          & 66              & 308          & 224\_708\_703\_047    & 0.99948          & 109             \\
        \hline
        217             & 96\_883\_183      & 0.99899          & 67              & 309          & 20\_136\_615\_963     & 0.99955          & 106             \\
        \hline
        218             & 1\_583\_507\_967  & 0.99875          & 79              & 310          & 199\_741\_069\_375    & 0.99937          & 110             \\
        \hline
        219             & 661\_398\_811     & 0.99972          & 73              & 311          & 1\_150\_284\_049\_727 & 0.99959          & 104             \\
        \hline
        220             & 1\_968\_165\_887  & 0.99995          & 82              & 312          & 23\_865\_618\_919     & 0.99942          & 107             \\
        \hline
        221             & 2\_079\_441\_767  & 0.99964          & 75              & 313          & 620\_398\_672\_495    & 0.99999          & 108             \\
        \hline
        222             & 326\_610\_023     & 0.99905          & 88              & 314          & 21\_213\_883\_483     & 0.99963          & 108             \\
        \hline
        223             & 984\_082\_943     & 0.99993          & 83              & 315          & 149\_805\_802\_031    & 0.99934          & 109             \\
        \hline
        224             & 1\_232\_261\_787  & 0.99961          & 75              & 316          & 766\_856\_033\_151    & 0.99950          & 105             \\
        \hline
        225             & 656\_055\_295     & 0.99993          & 86              & 317          & 1\_131\_779\_353\_631 & 0.99986          & 112             \\
        \hline
        226             & 2\_848\_461\_311  & 0.99983          & 68              & 318          & 99\_870\_534\_687     & 0.99937          & 110             \\
        \hline
        227             & 409\_344\_047     & 0.99999          & 90              & 319          & 478\_337\_265\_823    & 0.99977          & 114             \\
        \hline
        228             & 272\_896\_031     & 0.99999          & 92              & 320          & 566\_918\_240\_975    & 0.99973          & 115             \\
        \hline
        229             & 181\_930\_687     & 1.00000          & 93              & 321          & 425\_188\_680\_731    & 0.99971          & 115             \\
        \hline
        230             & 1\_304\_621\_055  & 0.99981          & 83              & 322          & 140\_284\_537\_063    & 0.99932          & 112             \\
        \hline
        231             & 1\_324\_921\_887  & 0.99734          & 73              & 323          & 188\_972\_746\_991    & 0.99974          & 121             \\
        \hline
        232             & 95\_592\_191      & 0.99992          & 95              & 324          & 251\_963\_662\_655    & 0.99971          & 117             \\
        \hline
        233             & 1\_104\_180\_463  & 1.00000          & 82              & 325          & 167\_975\_775\_103    & 0.99973          & 123             \\
        \hline
        234             & 5\_328\_487\_839  & 0.99997          & 80              & 326          & 3\_654\_218\_733\_311 & 0.99990          & 113             \\
        \hline
        235             & 13\_551\_207\_911 & 0.99998          & 82              & 327          & 125\_981\_831\_327    & 0.99971          & 122             \\
        \hline
        236             & 9\_034\_138\_607  & 0.99997          & 82              & 328          & 566\_619\_806\_719    & 0.99980          & 122             \\
        \hline
        237             & 63\_728\_127      & 0.99984          & 96              & 329          & 2\_253\_835\_349\_759 & 0.99974          & 105             \\
        \hline
        238             & 15\_218\_280\_607 & 1.00000          & 75              & 330          & 1\_325\_730\_144\_347 & 0.99997          & 115             \\
        \hline
        239             & 17\_108\_656\_891 & 0.99890          & 84              & 331          & 1\_649\_531\_356\_143 & 0.99990          & 108             \\
        \hline
    \end{tabular}
}

\noindent\makebox[\textwidth]{
    \begin{tabular}{|c c c c| c c c c|}
        \hline
        \textbf{Run}    &                    &                  & \textbf{Sample} & \textbf{Run} &                        &                  & \textbf{Sample} \\
        \textbf{Length} & \textbf{Seed}      & \textbf{Entropy} & \textbf{Size}   & \textbf{Run} & \textbf{Seed}          & \textbf{Entropy} & \textbf{Size}   \\
        \hline
        240             & 3\_246\_339\_311   & 0.99929          & 80              & 332          & 1\_176\_549\_020\_911  & 0.99970          & 122             \\
        \hline
        241             & 2\_164\_226\_207   & 0.99921          & 80              & 333          & 1\_287\_402\_586\_111  & 0.99945          & 108             \\
        \hline
        242             & 1\_442\_817\_471   & 0.99917          & 81              & 334          & 26\_130\_934\_783      & 0.99950          & 114             \\
        \hline
        243             & 20\_445\_954\_119  & 0.99967          & 73              & 335          & 1\_303\_333\_417\_199  & 0.99990          & 109             \\
        \hline
        244             & 13\_630\_636\_079  & 0.99952          & 73              & 336          & 83\_987\_887\_551      & 0.99973          & 123             \\
        \hline
        245             & 9\_087\_090\_719   & 0.99952          & 73              & 337          & 881\_989\_193\_575     & 0.99998          & 117             \\
        \hline
        246             & 6\_058\_060\_479   & 0.99950          & 74              & 338          & 20\_646\_664\_519      & 0.99957          & 115             \\
        \hline
        247             & 18\_019\_682\_047  & 0.99994          & 90              & 339          & 1\_267\_630\_141\_951  & 0.99994          & 116             \\
        \hline
        248             & 17\_825\_084\_863  & 0.99996          & 86              & 340          & 18\_352\_590\_683      & 0.99958          & 116             \\
        \hline
        249             & 217\_740\_015      & 0.99914          & 90              & 341          & 24\_470\_120\_911      & 0.99963          & 116             \\
        \hline
        250             & 1\_801\_487\_687   & 0.99891          & 83              & 342          & 883\_820\_096\_231     & 0.99998          & 115             \\
        \hline
        251             & 1\_200\_991\_791   & 0.99876          & 84              & 343          & 21\_751\_218\_587      & 0.99983          & 118             \\
        \hline
        252             & 16\_670\_963\_135  & 0.99965          & 75              & 344          & 12\_235\_060\_455      & 0.99973          & 119             \\
        \hline
        253             & 6\_250\_517\_663   & 0.99957          & 93              & 345          & 5\_086\_317\_509\_375  & 0.99979          & 127             \\
        \hline
        254             & 32\_060\_507\_419  & 0.99952          & 86              & 346          & 898\_696\_369\_947     & 0.99921          & 121             \\
        \hline
        255             & 14\_884\_335\_615  & 0.99982          & 92              & 347          & 18\_570\_171\_467\_519 & 0.99934          & 122             \\
        \hline
        256             & 87\_147\_171\_839  & 0.99957          & 89              & 348          & 5\_157\_142\_856\_607  & 0.99805          & 118             \\
        \hline
        257             & 6\_216\_083\_103   & 0.99957          & 81              & 349          & 4\_018\_818\_772\_839  & 0.99979          & 129             \\
        \hline
        258             & 8\_781\_412\_679   & 0.99977          & 79              & 350          & 5\_509\_607\_710\_143  & 0.99996          & 117             \\
        \hline
        259             & 24\_083\_989\_231  & 0.99918          & 85              & 351          & 10\_681\_465\_356\_287 & 0.99811          & 119             \\
        \hline
        260             & 23\_962\_604\_007  & 1.00000          & 92              & 352          & 7\_120\_976\_904\_191  & 0.99807          & 120             \\
        \hline
        261             & 21\_407\_990\_427  & 0.99914          & 86              & 353          & 4\_747\_317\_936\_127  & 0.99800          & 125             \\
        \hline
        262             & 5\_854\_275\_119   & 0.99974          & 80              & 354          & 7\_244\_052\_517\_375  & 0.99801          & 118             \\
        \hline
        263             & 3\_902\_850\_079   & 0.99960          & 82              & 355          & 19\_754\_675\_554\_139 & 0.99926          & 120             \\
        \hline
        264             & 349\_414\_071\_423 & 1.00000          & 89              & 356          & 13\_169\_783\_702\_759 & 0.99934          & 120             \\
        \hline
        265             & 25\_244\_554\_015  & 1.00000          & 94              & 357          & 17\_559\_711\_603\_679 & 0.99914          & 120             \\
        \hline
        266             & 81\_774\_557\_807  & 0.99981          & 90              & 358          & 12\_380\_114\_311\_679 & 0.99926          & 122             \\
        \hline
        267             & 60\_142\_063\_643  & 0.99941          & 88              & 359          & 8\_253\_409\_541\_119  & 0.99938          & 122             \\
        \hline
        268             & 4\_111\_644\_527   & 0.99947          & 84              & 360          & 18\_026\_976\_767\_615 & 0.99973          & 129             \\
        \hline
        269             & 5\_482\_192\_703   & 0.99940          & 83              & 361          & 5\_521\_395\_748\_159  & 0.99990          & 124             \\
        \hline
        270             & 2\_741\_096\_351   & 0.99950          & 86              & 362          & 8\_011\_989\_674\_495  & 0.99974          & 130             \\
        \hline
        271             & 23\_759\_827\_611  & 0.99944          & 89              & 363          & 4\_141\_046\_811\_119  & 0.99994          & 125             \\
        \hline
        272             & 139\_869\_168\_255 & 0.99948          & 94              & 364          & 7\_121\_768\_599\_551  & 0.99972          & 132             \\
        \hline
        273             & 1\_827\_397\_567   & 0.99953          & 86              & 365          & 31\_508\_135\_471\_707 & 0.99968          & 130             \\
        \hline
        274             & 28\_159\_795\_687  & 0.99936          & 89              & 366          & 2\_760\_697\_874\_079  & 0.99996          & 126             \\
        \hline
        275             & 59\_834\_174\_399  & 0.99830          & 92              & 367          & 33\_508\_530\_061\_951 & 0.99998          & 119             \\
        \hline
        276             & 4\_704\_765\_167   & 0.99948          & 96              & 368          & 18\_307\_067\_699\_951 & 0.99611          & 119             \\
        \hline
        277             & 6\_273\_020\_223   & 0.99944          & 96              & 369          & 12\_204\_711\_799\_967 & 0.99612          & 120             \\
        \hline
        278             & 39\_889\_449\_599  & 0.99845          & 94              & 370          & 2\_813\_538\_212\_167  & 0.99983          & 137             \\
        \hline
        279             & 53\_185\_932\_799  & 0.99837          & 94              & 371          & 23\_838\_942\_284\_287 & 0.99939          & 129             \\
        \hline
    \end{tabular}
}

\end{document}